\documentclass[a4paper,14pt]{article}
\fontfamily{Times New Roman}
%%% Работа с русским языком
% кодировка исходного текста
%\usepackage{l3backend}
%\usepackage{fontspec}{Times New Roman}

\usepackage{cmap}					% поиск в PDF
\usepackage[T2A]{fontenc}			% кодировка
\usepackage[utf8]{inputenc}			% кодировка исходного текста
\usepackage[english,russian]{babel}	% локализация и переносы
\usepackage[obeyspaces]{url}
\usepackage{graphicx} % Required for including pictures
\usepackage{microtype} % Improves typography
\usepackage[14pt]{extsizes}
\usepackage{multirow}
\usepackage{lscape}
%\usepackage{mathptmx}
%\userpackage{mathptmx}
%\userpackage{times}
%\usepackage[absolute]{textpos}

%%% Работа с русским языком
\usepackage{indentfirst}			% красная черта в первом абзаце
\frenchspacing


%% Свои команды
\newcommand{\mysection}[1]{\section*{#1} \addcontentsline{toc}{subsection}{#1}}
\newcommand{\mysubsection}[1]{\subsection*{#1} \addcontentsline{toc}{subsubsection}{#1}}


%%% Страница
\usepackage{geometry} % Простой способ задавать поля
\geometry{top=15mm}
\geometry{bottom=30mm}
\geometry{left=30mm}
\geometry{right=20mm}

\usepackage{setspace} % Интерлиньяж
%\onehalfspacing % Интерлиньяж 1.5
%\doublespacing % Интерлиньяж 2
\singlespacing % Интерлиньяж 1

\renewcommand{\labelenumii}{\arabic{enumi}.\arabic{enumii}}
\renewcommand{\labelenumiii}{\arabic{enumi}.\arabic{enumii}.\arabic{enumiii}}

\begin{document} % конец преамбулы, начало документа
	
	\begin{minipage}{0.4\textwidth}
	\includegraphics[scale=0.12]{Logo_DPSMain.png} 
	\end{minipage}
\hfill
\hfill
	\begin{minipage}{0.45\textwidth}
		\begin{center}
	Управління по роботі з податковим боргом ГУ ДПС в Івано-Франківській області
		\end{center}
   \end{minipage}

\bigskip

\hrule height 2pt\smallskip
\hrule height 0.5pt%
\bigskip
20.02.2024	
\begin{center}

	
\emph{\textbf{Інформаційна довідка}}\\
«Про результати роботи за напрямком «\underline{Скорочення податкового боргу}» в 2023-2024 роках»\\
\bigskip
\textbf{Динаміка податкового боргу }

2023-2024 рр.
	
\end{center}

\begin{figure}[h]
	\centering
	\includegraphics[width=1\linewidth]{"Динаміка податкового боргу помісячно"}
	\caption{Динаміка податкового боргу в 2023-2024 році}
	\label{fig:-1}
\end{figure}

В 2023-2024 роках має місце тенденція щодо скорочення податкового боргу, основною факторами впливу на неї є:
\begin{itemize}
	\item погашення податкового боргу по платі за надра (нафта, газ, конденсат) в січні - квітні 2023 року на суму 1,2 млрд.гривень;
	\item списання безнадійного податкового боргу, в тому числі по ліквідованих банкрутах в січні 2024 року на суму 230,3 млн.гривень
\end{itemize}

Епізодичне зростання податкового боргу в другій половині року пов'язано в основному із нарощенням податкового боргу по майнових податках - якщо в першому півріччі 2023 року величина податкового боргу по майнових податках зросла порівняно із початком року на 19,1 млн. гривень або на 3,8 відс., то у другому півріччі порівняно із 01 липня  2023 р. - на 93,4 млн. гривень або на 18,1 відсотків.

\begin{figure}[h]
	\centering
	\includegraphics[width=1\linewidth]{"Динаміка кількості боржників помісячно"}
	\caption{Динаміка кількості боржників в 2023-2024 році}
	\label{fig:-2}
\end{figure}

Щодо кількості боржників, то тенденція щодо їх зменшення в результаті планомірної роботи по погашенню (скороченню) податкового боргу в 2023 році декілька раз протягом року (лютий 2023, серпень 2023, грудень 2023, січень 2024) змінювала свій напрямок, що пов'язано із періодами проведення нарахувань по майнових податках та МПЗ фізичних осіб - громадян.\\

Найбільшу частку в кількості боржників займають боржники по майнових податках (юридичні та фізичні особи) - 227136 з 243622 (або 93,2 відсотків) є боржниками по майнових податках. Їх кількість збільшилась з початку року з 188251 на початок року до 230329 станом на 01.01.2024 (або на 22 відс.). Разом із тим, у періоди планомірної роботи по їх зменшенню їх кількість зменшуєть в середньому на місяць на 4,4 тис. осіб або на 2,2 відс.\\

Також, протягом 2023-2024 років забезпечено скорочення заборгованості (недоїмки) по єдиному соціальному внеску. Так, його величина станом на 01 лютого 2024 року становить 164,9 млн. гривень, що на 44,4 млн. гривень або на 21,2 відс. менше величини на початок цього року. \\


\begin{figure}[h]
	\centering
	\includegraphics[width=1\linewidth]{"Динаміка кількості боржників та заборгованості ЄВ"}
	\caption{Динаміка кількості боржників та заборгованості ЄВ в 2023-2024 році}
	\label{fig:-3}
\end{figure}

Аналогічно, забезпчено скорочення кількості боржників по ЄСВ - їх кількість станом на 01.02.2024 року становить 7833, що менше на 2045 або на 20,7 відс., ніж на початок року.\\

В 2023 році забезпечено виконання доведених показників доходів за напрямком роботи  "Робота з податковим боргом":
\begin{itemize}
	\item забезпечення надходжень до Державного бюджету - 281 відс. (план - 80 млн. грн., факт - 225 млн. грн.);
	\item забезпечення надходжень в рахунок погашення боргу по ЄВ - 172,7 відс. (план - 28,4 млн. грн., факт - 49,0 млн. грн.);
	\item забезпечення надходжень за рахунок реалізації майна боржників - 32,3 відс. (план - 2,4 млн. грн., факт - 0,8 млн. грн.);
	\item забезпечення надходжень по безхозу, зокрема до ДБ - 184,3 відс. (план - 99 тис.грн., факт - 182,4 тис. грн.), до МБ - 790 відс. (план - 107 тис. грн., факт - 845,6 тис. грн.) 
\end{itemize} 


В січні 2024 року забезпечено виконання доведених показників доходів за напрямком роботи  "Робота з податковим боргом":
\begin{itemize}
	\item забезпечення надходжень до Державного бюджету - 102,5 відс. (план - 4,2 млн. грн., факт - 4,3 млн. грн.);
	\item забезпечення надходжень в рахунок погашення боргу по ЄВ - 110,5 відс. (план - 2,1 млн. грн., факт - 2,3 млн. грн.);
	\item забезпечення надходжень за рахунок реалізації майна боржників - 0 відс. (план - 30 тис. грн., факт - 0 тис.грн.);
	\item забезпечення надходжень по безхозу, зокрема до ДБ - 57 відс. (план - 12 тис.грн., факт - 6,8 тис. грн.), до МБ - 41,5 відс. (план - 8 тис. грн., факт - 3,3 тис. грн.) 
\end{itemize} 

В лютому 2024 року станом на 20 лютого забезпечено виконання доведених показників доходів за напрямком роботи  "Робота з податковим боргом":
\begin{itemize}
	\item забезпечення надходжень до Державного бюджету - 35,6 відс. (план - 3,8 млн. грн., факт - 1,4 млн. грн.), очікується рівень виконання 100 відс.;
	\item забезпечення надходжень в рахунок погашення боргу по ЄВ - 69,6 відс. (план - 2,2 млн. грн., факт - 1,6 млн. грн.), очікується виконання на 105 відс.;
	\item забезпечення надходжень по безхозу, зокрема до ДБ - 6,5 відс. (план - 15 тис.грн., факт - 1 тис. грн.), до МБ - 14,1 відс. (план - 7 тис. грн., факт - 1 тис. грн.) 
\end{itemize} 
	
	
	
\end{document}